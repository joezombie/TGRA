\documentclass[12pt]{article}

\usepackage[english]{babel}
\usepackage[utf8]{inputenc}
\usepackage[T1]{fontenc}

\usepackage{geometry}
\geometry{a4paper}

\usepackage{xspace}

\usepackage{graphicx}
\usepackage{enumerate}
\usepackage{verbatim}

\usepackage{float}
\usepackage{wrapfig}
\usepackage{hyperref}
\usepackage{scrextend}
\usepackage{indentfirst}
\linespread{1.2}

\usepackage{color}
\definecolor{darkblue}{rgb}{0,0,0.5}
\definecolor{darkred}{rgb}{0.5,0,0}

\hypersetup{
    colorlinks=true,
    linkcolor=darkblue
}


\graphicspath{{pics/}}



\begin{document}
	\newcommand{\tr}{Tracker\xspace}

	{\huge Lýsing á appi\\}
	
	Appið gerir notandanum kleift að búa til áminningar sem að birtist þegar hann kemur á ákveðinn stað eða þegar 
	hann fer frá ákveðnum stað.\\
	\\
	Notandinn getur vistað staðsetningar sem að hann hefur áhuga á, til dæmis heima hjá sér eða á vinnustaðnum sínum.
	Það er gert með því að velja annaðhvort núverandi staðsetningu eða velja staðsetningu á korti.\\
	\\
	Notandinn getur búið til skilaboð, valið staðsetningu og valið að láta birta skilaboðin næst þegar hann annaðhvort kemur á staðsetninguna eða næst þegar hann yfirgefur staðsetninguna.\\
	\\
	\\
	Jóhannes Gunnar Heiðarsson

\end{document}
